% Metódy inžinierskej práce

\documentclass[10pt,twoside,slovak,a4paper]{article}

\usepackage[slovak]{babel}
\usepackage[IL2]{fontenc}
\usepackage[utf8]{inputenc}
\usepackage{graphicx}
\usepackage{url} 
\usepackage{hyperref} 

%\usepackage{cite}

\graphicspath{{./}}

\pagestyle{headings}

\title{Dizajn a modelovanie užívateľských rozhraní pre ľudí so zdravotnými postihnutiami\thanks{Semestrálny projekt v predmete Metódy inžinierskej práce, ak. rok 2021/22, vedenie: Ing. Vladimír Mlynarovič, PhD.}} % meno a priezvisko vyučujúceho na cvičeniach

\author{Martin Mislovič\\[2pt]
	{\small Slovenská technická univerzita v Bratislave}\\
	{\small Fakulta informatiky a informačných technológií}\\
	{\small \texttt{xmislovic@stuba.sk}}
	}

\date{\small 11. október 2021}



\begin{document}

\maketitle

\begin{abstract}
V dnešnej dobe existuje veľa metód a techník ako správne postupovať pri dizajne a modelovaní užívateľských rozhraní.
Mnoho z nich však zabúda na to, že dané užívateľské rozhrania budú používať aj užívatelia s rôznymi zdravotnými postihnutiami, 
pre ktorých môže byť používanie klasických užívateľských rozhraní komplikované, až nemožné. Tento článok chce priblížiť v 
čom sú nároky užívateľov so zdravotnými postihnutiami iné od nárokov bežných užívateľov, aké sú rozdiely v modelovaní bežných 
užívateľských rozhraní a užívateľských rozhraní pre zdravotne postihnutých, aké nástroje sú používané pri takomto modelovaní 
a aké techniky a pravidlá sa pri tvorbe, dizajne a modelovaní takýchto užívateľských rozhraní uplatňujú.
\end{abstract}

\begin{figure}[h]
	\centering
	\includegraphics[width=\textwidth]{diagram-crop.pdf}
	\caption{Vpliv rozhraní nevidomých na HCI modely}
\end{figure}


%\bibliography{literatura}
%\bibliographystyle{plain} % prípadne alpha, abbrv alebo hociktorý iný
\end{document}
