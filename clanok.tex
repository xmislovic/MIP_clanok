\documentclass[10pt,twoside,slovak,a4paper]{article}

\usepackage[slovak]{babel}
\usepackage[IL2]{fontenc}
\usepackage[utf8]{inputenc}
\usepackage{graphicx}
\usepackage{url} 
\usepackage{hyperref} 

\usepackage{cite}

\graphicspath{{./}}

\pagestyle{headings}

\title{Dizajn a modelovanie užívateľských rozhraní pre ľudí so zdravotnými postihnutiami \thanks{Semestrálny projekt v predmete Metódy inžinierskej práce, ak. rok 2021/22, vedenie: Ing. Vladimír Mlynarovič, PhD.}} % meno a priezvisko vyučujúceho na cvičeniach

\author{Martin Mislovič\\[2pt]
	{\small Slovenská technická univerzita v Bratislave}\\
	{\small Fakulta informatiky a informačných technológií}\\
	{\small \texttt{xmislovic@stuba.sk}}
	}

\date{\small 11. október 2021}



\begin{document}

\maketitle

\begin{abstract}
% Definitívne treba prerobiť
V dnešnej dobe existuje veľa metód a techník ako správne postupovať pri dizajne a modelovaní užívateľských rozhraní.
Mnoho z nich však zabúda na to, že dané užívateľské rozhrania budú používať aj užívatelia s rôznymi zdravotnými postihnutiami, 
pre ktorých môže byť používanie klasických užívateľských rozhraní komplikované, až nemožné. Tento článok chce priblížiť v 
čom sú nároky užívateľov so zdravotnými postihnutiami iné od nárokov bežných užívateľov, aké sú rozdiely v modelovaní bežných 
užívateľských rozhraní a užívateľských rozhraní pre zdravotne postihnutých, aké nástroje sú používané pri takomto modelovaní 
a aké techniky a pravidlá sa pri tvorbe, dizajne a modelovaní takýchto užívateľských rozhraní uplatňujú.
\end{abstract}

\section{Modelovanie užívatelských rozhraní}

\section{Nároky zdravotne postihnutých užívateľov}

\subsection{Ako s počítačmi interagujú zdravotne postihnutý}
V dnešnej dobe takmer každý interaguje v nejakej forme s počítačom pomocou užívalských rozhraní.
To platí aj pre ľudí s rôznymimi zdravotnými postihnutiamy. Boli vyvinuté rôzne asistenčné
technológie, ktoré umožňujú nevidiacim, zle vidiacim, ľudom s dyslexiou či inými poruchami
spracovávania, hluchým, zle počujúcim či ľudom bez rúk interagovať s počítačom. 
Na to aby takéto asistenčné technológie fungovali je však potrebné aby tieto Uživatelské
spĺňali určité podmienky\cite{10.1145/2559206.2567823}.

\subsection{Príklad nárokov zdravotne postihnutých užívateľov}
Nároky na užívatelské rozhrania sa menia podľa zdravotného postihnutia a teda druhu asistenčnej
technológie. Hlavné myšlienky a základné princípy s, ktorými treba pristupovať ku tvorbe takýchto
rozhraní sú však podobné. Ako príklad nárokov zdravotne postihnutých uvediem nároky slepých ľudí.
Tie sa dajú rozdeliť do piatich kategórií:
\begin{enumerate}
	\item \textbf{Adekvátnosť} - Úloha musí byť adekvátna schopnostiam slepých uživateľov
	\item \textbf{Dimenzionálny kompromis} - Užívateľské rozhranie musí zachovať balanc
	medzi 2D prístupom vidiacih a 1D prístupom slepých
	\item \textbf{Rovnosť správania} - Slepý užívatelia by mali mať prístup ku všetkým relevantným
	častiam uživateľského rozhrania
	\item \textbf{Prevensia semantickej straty} - Užívatelské rozhranie musí predísť strate
	semantických informácií
	\item \textbf{Nezávislosť na zariadení} - Uživatelské rozhranie by malo fungovať na rôznych
	asistenčných technológiach
\end{enumerate}
Tieto nároky majú dopad na všetky modely použité vo vývoji
"human-computer" rozhraní.\cite{10.1007/978-3-540-70540-6_117}
\begin{figure}[ht]
	\centering
	\includegraphics[width=\textwidth]{diagram-crop.pdf}
	\caption{Vpliv nárokov nevidomých na HCI modely\cite{10.1007/978-3-540-70540-6_117}}
\end{figure}

\section{Vplyv nárokov zdravotne postihnutých na jednotlivé modely}

\section{Nástroje pre tvorbu užívalských rozhraní pre zdravotne postihnutých}

\section{Zásady prístupného dizajnu}

\bibliography{literatura}
\bibliographystyle{plain} % prípadne alpha, abbrv alebo hociktorý iný
\end{document}
