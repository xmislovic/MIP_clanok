\documentclass[10pt,twoside,slovak,a4paper]{article}

\usepackage[slovak]{babel}
\usepackage[IL2]{fontenc}
\usepackage{tabularx}
\usepackage[utf8]{inputenc}
\usepackage{graphicx}
\usepackage{url} 
\usepackage{hyperref} 

\usepackage{cite}

\graphicspath{{./}}

\pagestyle{headings}

\title{Dizajn a modelovanie užívateľských rozhraní pre ľudí so zdravotnými postihnutiami \thanks{Semestrálny projekt v predmete Metódy inžinierskej práce, ak. rok 2021/22, vedenie: Ing. Vladimír Mlynarovič, PhD.}} % meno a priezvisko vyučujúceho na cvičeniach

\author{Martin Mislovič\\[2pt]
	{\small Slovenská technická univerzita v Bratislave}\\
	{\small Fakulta informatiky a informačných technológií}\\
	{\small \texttt{xmislovic@stuba.sk}}
	}

\date{\small 11. október 2021}



\begin{document}

\maketitle

\begin{abstract}
% Definitívne treba prerobiť
V dnešnej dobe existuje veľa metód a techník ako správne postupovať pri dizajne a modelovaní užívateľských rozhraní.
Mnoho z nich však zabúda na to, že dané užívateľské rozhrania budú používať aj užívatelia s rôznymi zdravotnými postihnutiami, 
pre ktorých môže byť používanie klasických užívateľských rozhraní komplikované, až nemožné. Tento článok chce priblížiť v 
čom sú nároky užívateľov so zdravotnými postihnutiami iné od nárokov bežných užívateľov, aké sú rozdiely v modelovaní bežných 
užívateľských rozhraní a užívateľských rozhraní pre zdravotne postihnutých, aké nástroje sú používané pri takomto modelovaní 
a aké techniky a pravidlá sa pri tvorbe, dizajne a modelovaní takýchto užívateľských rozhraní uplatňujú.
\end{abstract}

\section{Modelovanie užívatelských rozhraní}
\subsection{HCI}
Užívatelské rozhrania a to ako s nimi ľudia interagujú sú komplexné témy, ktorými sa zaoberá viacero vedeckých oborov.
Spoločné pomenovanie pre toto vedecké pole je HCI - human computer interactions (interakcie človeka s počítačom).
\section{Nároky zdravotne postihnutých užívateľov a ich dopad na tvorbu UR}
\subsection{Ako s počítačmi interagujú zdravotne postihnutý}
V dnešnej dobe takmer každý interaguje v nejakej forme s počítačom pomocou užívalských rozhraní. Pre život v modernej
spoločnosti je počítač a schopnosť interagovať s ním nevyhnutnosť. To platí aj pre ľudí s rôznymimi zdravotnými
postihnutiamy. Pre tých však bývajú tieto interakcie mnohokrát náročnejšie ako pre bežných ľudí. Z toho dôvodu boli 
vyvinuté rôzne asistenčné technológie, ktoré umožňujú nevidiacim, zle vidiacim, ľudom s dyslexiou či inými poruchami
spracovávania, hluchým, zle počujúcim či ľudom bez rúk interagovať s počítačom. Príkladom môžu byť čítačky obrazovky
využívané slepými alebo nástroje ako "The Easier system" ktoré, pomocou  kombinácie metód zo spracovávania prirodzeného
jazyka a HCI vedia zjednodušiť text pre ludí s kognitívnymi poruchami\cite{10.1145/3471391.3471400}.Na to aby takéto
asistenčné technológie fungovali je však potrebné aby obsah ale i samotné médium podávajúce tento obsah spĺňali určité
podmienky\cite{10.1145/2559206.2567823}.

\subsection{Príklad nárokov zdravotne postihnutých užívateľov}
Nároky na užívatelské rozhrania sa menia podľa zdravotného postihnutia a teda druhu asistenčnej
technológie. Hlavné myšlienky a základné princípy s, ktorými treba pristupovať ku tvorbe takýchto
rozhraní sú však podobné. Ako príklad nárokov zdravotne postihnutých uvediem nároky slepých ľudí.
Tie sa dajú rozdeliť do piatich kategórií:
\begin{enumerate}
	\item \textbf{Adekvátnosť} - Úloha musí byť adekvátna schopnostiam slepých uživateľov
	\item \textbf{Dimenzionálny kompromis} - Užívateľské rozhranie musí zachovať balanc
	medzi 2D prístupom vidiacih a 1D prístupom slepých
	\item \textbf{Rovnosť správania} - Slepý užívatelia by mali mať prístup ku všetkým relevantným
	častiam uživateľského rozhrania
	\item \textbf{Prevensia semantickej straty} - Užívatelské rozhranie musí predísť strate
	semantických informácií
	\item \textbf{Nezávislosť na zariadení} - Uživatelské rozhranie by malo fungovať na rôznych
	asistenčných technológiach
\end{enumerate}
Tieto nároky majú dopad na všetky modely použité vo vývoji
"human-computer" rozhraní.\cite{10.1007/978-3-540-70540-6_117}
\begin{figure}[h]
	\centering
	\includegraphics[width=\textwidth]{diagram-crop.pdf}
	\caption{Vpliv nárokov nevidomých na HCI modely\cite{10.1007/978-3-540-70540-6_117}}
\end{figure}
\subsection{Vplyv nárokov zdravotne postihnutých na jednotlivé modely}

\section{Pravidlá pri tvorbe prístupných užívatelských rozhraní}
Kvôli dôležitosti prístupnosti užívatelských rozhraní pre všetkých užívateľov bolo vyvynutých niekoľko sád pravidiel a
usmernení. V dnešnej dobe sú stránky ako eHealth či eGoverment viac a viac populárne, niet teda divu, že väčšina z 
týchto pravidiel a regulácií sa zameriava práve na internet a prístupnosť webstránok. Najrozšírenejšie z týchto
usmernení sú "The Web Content Accessibility Guidelines (WCAG)" vytvorené W3C (World Wide Web Consortium)\cite{10.1145/3471391.3471400}. Tie sú
postavené na štyroch princípoch: vnímatelnosť, ovladateľnosť, porozumitelnosť a robustnosť
\cite{kirkpatrick_connor_campbell_cooper_2018}.

\begin{table}[h]
	\begin{tabularx}{\textwidth}{|X|X|}
		\hline
		\textbf{Textové alternatívy} & Netextový obsah by mal mať textovú náhradu \\
		\hline
		\textbf{Časovo založené média} & Mali by byť ponúknuté alternatívy k časovo založeným médiám \\
		\hline
		\textbf{Prisposobivosť} & Obsah by malo byť možné prezentovať rôznymi spôsobmi bez straty informácií či štruktúry \\
		\hline
		\textbf{Rozoznatelnosť} & Malo by by jednoduché rozlíšiť popredie od pozadia \\
		\hline
		\textbf{Klávesnica} & Všetka funkcionalita by mala byť dostupná pomocou klávesnice \\
		\hline
		\textbf{Poskytnúť dosť času} & Užívateľ by mal ma dosť času na čítanie a používanie obsahu\\
		\hline
		\textbf{Záchvaty a fyzické reakcie} & Obsah by nemal spôsobovať záchvaty či iné fyzické reakcie \\
		\hline
		\textbf{Navigovatelný} & Užívateľ by maľ mať možnosť navigovať sa, nájsť obsah a zistiť kde je \\
		\hline
		\textbf{Modalita vstupu} & Malo by byt jednoduché ovládať funkcionalitu aj pomocou iných metód ako klávesnica \\
		\hline
		\textbf{Čitatelnosť} & Obsah by mal byť čitatelný a zrozumitelný \\
		\hline
		\textbf{Predvídatelnosť} & Web stránky by sa mali správať predvídatelným spôsobom \\
		\hline
		\textbf{Asistencia vstupu} & Web stránka by mala pomáhať užívateľom predísť chybám \\
		\hline
		\textbf{Kompatibilita} & Obsah by mal byť kompatibilný so súčasnými i budúcimi asistenčnými technológiami \\
		\hline
	\end{tabularx}
	\caption{Usmernenia podľa aktuálneho znenia WCAG \cite{kirkpatrick_connor_campbell_cooper_2018}}
\end{table}

I keď sú tieto usmernenia najrozšírenejšie, neznamená to, že ich dodržaním vznikne webstránka vhodná pre všetkých. Štúdie
ukázali, že užívatelia s kognitívnymi postihnutiami mali problém aj pri užívaní webstránok spĺňajúcich usmernenia WCAG
\cite{10.1145/3471391.3471400}. Z toho dôvodu vznikli ďalšie usmernenia špecificky pre takýchto používateľov. Príkladom
takýchto usmernenení môžu byť "The Easy to Read guidelines" či "European Guidelines for the Production of Easy-to-Read
information". I keď je asi nemožné aby webstránka spĺňala každé usmernenie je dôležité a občas dokonca zákonom vyžadované
aby sa ich pokúšala splniť čo najviac.

\section{Nástroje pre tvorbu užívalských rozhraní pre zdravotne postihnutých}

\bibliography{literatura}
\bibliographystyle{plain} % prípadne alpha, abbrv alebo hociktorý iný
\end{document}
