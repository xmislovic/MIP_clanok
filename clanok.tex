\documentclass[10pt,twoside,slovak,a4paper]{article}

\usepackage[slovak]{babel}
\usepackage[IL2]{fontenc}
\usepackage[utf8]{inputenc}
\usepackage{graphicx}
\usepackage{url} 
\usepackage{hyperref} 

%\usepackage{cite}

\graphicspath{{./}}

\pagestyle{headings}

\title{Dizajn a modelovanie užívateľských rozhraní pre ľudí so zdravotnými postihnutiami \thanks{Semestrálny projekt v predmete Metódy inžinierskej práce, ak. rok 2021/22, vedenie: Ing. Vladimír Mlynarovič, PhD.}} % meno a priezvisko vyučujúceho na cvičeniach

\author{Martin Mislovič\\[2pt]
	{\small Slovenská technická univerzita v Bratislave}\\
	{\small Fakulta informatiky a informačných technológií}\\
	{\small \texttt{xmislovic@stuba.sk}}
	}

\date{\small 11. október 2021}



\begin{document}

\maketitle

\begin{abstract}
% Definitívne treba prerobiť
V dnešnej dobe existuje veľa metód a techník ako správne postupovať pri dizajne a modelovaní užívateľských rozhraní.
Mnoho z nich však zabúda na to, že dané užívateľské rozhrania budú používať aj užívatelia s rôznymi zdravotnými postihnutiami, 
pre ktorých môže byť používanie klasických užívateľských rozhraní komplikované, až nemožné. Tento článok chce priblížiť v 
čom sú nároky užívateľov so zdravotnými postihnutiami iné od nárokov bežných užívateľov, aké sú rozdiely v modelovaní bežných 
užívateľských rozhraní a užívateľských rozhraní pre zdravotne postihnutých, aké nástroje sú používané pri takomto modelovaní 
a aké techniky a pravidlá sa pri tvorbe, dizajne a modelovaní takýchto užívateľských rozhraní uplatňujú.
\end{abstract}

\section{Modelovanie užívatelských rozhraní}

\section{Nároky zdravotne postihnutých užívateľov}
Uživatelia so zdravotnými postihnutiami majú odlišné nároky ako bežný uživatelia.
Rozdielnosť týchto nárokov sa odráža v tom ako treba pristupovať k modelovaniu
uživatelských rozhraní pre týchto ľudí. Nároky sa rôzňa podla druhu zdravotného
postihnutia no základná myšlienka zostáva rovnaká. Ako príklad uvediem nároky
slepých ľudí. Nároky slepých sa dajú rozdeliť do piatich kategórií:

\begin{enumerate}
	\item \textbf{Adekvátnosť} - Úloha musí byť adekvátna schopnostiam slepých uživateľov
	\item \textbf{Dimenzionálny kompromis} - Užívateľské rozhranie musí zachovať balanc
	medzi 2D prístupom vidiacih a 1D prístupom slepých
	\item \textbf{Rovnosť správania} - Slepý užívatelia by mali mať prístup ku všetkým relevantným
	častiam uživateľského rozhrania
	\item \textbf{Prevensia semantickej straty} - Užívatelské rozhranie musí predísť strate
	semantických informácií
	\item \textbf{Nezávislosť na zariadení} - Uživatelské rozhranie by malo fungovať na rôznych
	asistenčných technológiach
\end{enumerate} 
\begin{figure}[ht]
	\centering
	\includegraphics[width=\textwidth]{diagram-crop.pdf}
	\caption{Vpliv nárokov nevidomých na HCI modely}
\end{figure}

\section{Nástroje pre tvorbu užívalských rozhraní pre zdravotne postihnutých}

\section{Zásady prístupného dizajnu}

%\bibliography{literatura}
%\bibliographystyle{plain} % prípadne alpha, abbrv alebo hociktorý iný
\end{document}
